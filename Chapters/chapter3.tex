%!TEX root = ../template.tex
%%%%%%%%%%%%%%%%%%%%%%%%%%%%%%%%%%%%%%%%%%%%%%%%%%%%%%%%%%%%%%%%%%%%
%% chapter3.tex
%% NOVA thesis document file
%%
%% Chapter with a short laext tutorial and examples
%%%%%%%%%%%%%%%%%%%%%%%%%%%%%%%%%%%%%%%%%%%%%%%%%%%%%%%%%%%%%%%%%%%%
\chapter{Literature Review}
\label{cha:lit_review}

In this chapter, we explore the current state of the art and examine existing research that can be related to Non-verbal Communication and Physiological Data. We divide the research topics into the following areas, namely: the social impact of sharing physiological activity in human-human interactions, the appropriation of biosignals in creative practice and performance contexts, and methods for inferring emotional states from physiological data. After presenting a comprehensive collection of studies, we articulate our overall perspectives, justifying the intersection of these topics and proposing the design principles that manifest in the research outcomes described in the following chapters \ref{cha:preliminary Actions} and \ref{cha:case_studies}.

\section{Social Signals (WIP)}
People have a limited ability to (correctly) assess other people’s mental states (i.e., mentalizing) and interpret social cues, which is vital for social behavior [Polosan et al., 2011]. [Fernández et al., 2013; Howell et al., 2016]—Studies exemplify how one can amplify social cues and/or people’s sensitivity towards such cues based on foreign-to-self signals. In this vein, studies provide live biofeedback as a means to convey social cues in order to facilitate the mentalizing process and foster social interaction. Building on this theoretical pathway, studies investigate foreign live biofeedback as a driver for social interaction (Howell et al., 2018), social connectedness (Slovák et al., 2012), social experience (Mueller et al. 2010), social engagement (Snyder et al., 2015), or social support (Walmink et al., 2013). Studies often implement live biofeedback systems for social interaction by using ambient or wearable devices. Such devices facilitate foreign and self live biofeedback at the same time because users (both self and other) can potentially perceive them. Gervais et al. (2016) found that ambient live biofeedback devices can ease social interaction, foster empathy and relaxation, and promote self-reflection (Gervais et
al., 2016). Järvelä et al. (2016) report increased heart rate synchrony for dyads at different geographical locations. Roseway et al. (2015) report that their BioCrystal system resulted in users’ more highly recognizing their physiological states and supported interpersonal communication (Roseway et al., 2015). Slovák et al. (2012) found that heart rate sharing does not improve feelings of closeness in the workplace.

With Howell et al.’s (2016) wearable foreign live biofeedback t-shirt, pairs of friends can share emotions, such as joy or embarrassment. Wearable live biofeedback systems support users to enact social performances such as emotional engagement. Due to the ambiguity of the feedback, the authors found that the device often led to conversations about the wearer’s feelings. Generally, when providing ambient or wearable live biofeedback for social interaction, one  needs to consider users’ willingness to share their private physiological information 

\section{Communication Strategies with Physiological Activity}
\label{lit_review:biosignals_sharing}

There exist a number of studies that use heart rate sharing applications to enhance remote presence, but it's not well understood how this signal can be interpreted by a foreign body. In \cite{slovak_understanding_2012}, Slovak et al. investigate how people make sense of the heart rate feedback from other users, comparing visual and aural systems. The possible interpretations are classed as HR as information and HR as connection. The paper shows that auditory feedback of the cardiovascular activity can support connectedness between remote users without the need for additional context.  

\begin{center}
\begin{tabular}{|p{13cm}}
“with HR as information, participants consistently said that there was a need for context in order for the HR to be informative.  In contrast, with HR as connection, participants suggested that less context was better." 
\end{tabular}
\end{center}

...Greater distance actually improved feeling of connectedness, suggesting a equilibrium between the exchange informational content and connection 

Where a lot of the work shows success in improving social connectedness, we are also interested in cases in which this data allows subjects to convey emotional states. This work exemplifies how users can infer emotional traits from visual representations of another user's heart rate when supplemented with the events of a board game as context \cite{frey_remote_2016}. During gameplay, participants associated changes in other player's heart rate with bluffing, being upset, stress and even happiness. 

An extensive collection of studies under this topic have been reviewed by Ewa Lux et al. \cite{lux_live_2018}. The article lists 20 use-cases that apply what's described as \textit{foreign live biofeedback}, where user's are provided real-time feedback on another person's physiological state. For each use-case, the authors state the main focus, that being user experience, social interaction or stress management, the physiological modality utilised and the manifestation of the feedback, classed as visual, auditory, haptic or even as game mechanics.  

\section{Biosignals in Creative Practice}
\label{lit_review:biosignals_creativity}

The integration of biosignals in performance provides motivation to present physiological activity in creative ways so audience members can empathise with the performer by being exposed to their inner state. In this subsection, we review some examples that achieve this by utilising the aesthetic value of physiological sensor data to convey sort of emotional meaning. 

By considering the significance of aesthetics when presenting physiological activity, we are able to proceed with our underlying research goal of supporting non-verbal communication strategies. In section 1.XX, we highlight the relevance of non-verbal communication in therapeutic settings, so in addition to performance contexts, we are also interested in arts-therapy practices, which also rely on aesthetic engagement. In the article \cite{samaritter_aesthetic_2018}, Samaritter assess the impact music therapy and dance movement therapy can have on one's mental well-being, recognising the sensory-expressive experience of the maker as oppose to an audience.  Five core themes are derived from a literature review and comments taken from experts in the field, which are briefly as follows:

\begin{itemize}
\item Arts support and require embodied presence
\item Arts-therapy appeal to somatic resources, supporting anatomical, visceral and neuropsychological functioning
\item Arts support articulation and expression of emotional content.
\item Arts support sensitivity to non-verbal communication between participants. Furthermore, enactive empathy can be considered a core aspect arts-therapy practices.
\item The non-verbal or pre-verbal character of aesthetics was considered to support diversity and cross-cultural interaction
\end{itemize}

While experiencing a performance, and the audience is engaged, the performer-audience relationship becomes a collaborative process, because it is that role of the perceive to make any meaning.

\section{Psychological Effects of Abstract Representations (Brief Relevance to Art History)}
\label{lit_review:psychology_aesthetics}

\begin{itemize}
\item Going beyond the sensory enjoyment of art
\item fMRI study comparing representational and abstract, evoking different mental patterns from the perceiver
\item Triggering of psychological distance, relevance to emotional HCI and sharing biodata
\end{itemize}

As it will come apparent throughout the thesis document, a substantial part of this work is devoted to the integration of abstract art in the process of understanding and controlling our emotional functions, whether that be visual, sonic, material or otherwise. In parallel, we are taking insight from the non-classical view of emotion derived from Lisa Feldman Barrett's Research (See section X.X). Both of these concepts highlight the role of the perceiver to induce meaningful inference. Our objective is to correlate these ideas, presenting a continuous narrative between the different subject areas and consolidate the key themes in our design principles.  

In the 2013 journal article \textit{Two Modernist Approaches to Linking Art and Science}, Eric R. Kandel ties relevance to the art history concept of the beholder's share to the biological understanding of the human mind \cite{kandel_two_2013}. To articulate this methodology, Kandel focuses on a specific period within Viennise art culture, and compares two modernists ... and .... Where the former is focused more on the artist's developmental psychology, the article highlights the importance of aesthetic reception, suggesting that art is incomplete without the perceptual and emotional involvement of the viewer \cite{riegl_group_1999}. Riegal identified this notion at the \textit{beholder's share} (previously established as the beholder's involvement), in which the perceiver unconsciously assigns their own meaning to the non-representation in accordance to their previous life experiences, stimulating an emotional dialogue. This essentially of a viewer's perception was further explored by art historians Ernst Kris and E.H. Gombrich, defending the idea that to produce something personal, it is inherently ambiguous [Kris, E., \& Kaplan, a. (1952). aesthetic ambiguity. in E. Kris, Psychoanalytic explorations in art (pp. 243–264). New York: international Universities Press.]. 
In review of the studies referenced in the previous paragraphs, we can conclude that non-representations and aesthetics can engage the brain in new ways, developing new cognitive and emotional associations. {\color{red}expand}. By acknowledging this sentiment, we encourage the incorporation of abstract representations in Affective Computing research. Not only to allow flexible interpretations while a subject are engaged with the system, but also with the interest of supporting their mental well-being in the long term. By exposing users to abstract representations, it's possible that they become more accustom to sensing undefinable feelings in their everyday lives \cite{durkin_objective_2020}. With an alternative approach in mind, we can start to move away from goal-oriented emotion detection tasks and instead, prepare experiences that subtly adapt the user's cognitive behaviours over time. 

\newpage

\begin{figure}[htbp]
	\centering
	\includegraphics[width=0.8\textwidth]{Chapters/Figures/Concept_Venn.png}
	\caption{Intersection of Topics}
	\label{fig:Concept_Venn}
\end{figure}

% TODO:
% \begin{itemize}
% \item Briefly explain how each topic influences on another
% \item Include any cross-references, or identify where relevant cross-references are lacking
% \item End with reference and overview of phenomenology to consolidate themes. \textit{Investigations Into the Phenomenology and the Ontology of the Work of Art: What are Artworks and How Do We Experience Them?}: \url{ https://link.springer.com/content/pdf/10.1007%2F978-3-319-14090-2.pdf}
% \end{itemize}

\section{Affective Computing: Inferring Emotional states from Physiological Data}
\label{affective_computng_lit_review}

In this subsection, we will summarise the common practices in Affective Computing research over the past two decades and how this has evolved to the current state-of-the-art. 

The articles discussed in the following paragraphs should be used as a reference to the typical assumptions of the field, as for the rest of the thesis, we will be receptive to alternative views for designing emotionally informed systems. These are explained in more detail in section \ref{lit_review:conclusion} as we begin to deconstruct some of these assumptions in order to map out our methods. 

here is room for an affective computing that does not look at the body as “an instrument or object for the mind, passively receiving sign and signals, but not actively being part of producing them”—as phrased by Höök when referring to dominant paradigms in commercial sports applications [22]. However, how to best address bodily movement and engagement beyond measuring cues and signals is unclear. Most studies in affective computing revolve around affect recognition from emotion detection and bodily data classification \cite{bota_review_2019}. Research in this field commonly revolves around the detection of four basic emotions: fear, anger, sadness, and joy \cite{picard_mit_nodate}.

\textit{Conclude subsection justifying why interactional approach may be more suitable for interpersonal communication}

\begin{center}
\begin{tabular}{|p{13cm}}
\textbf{Cognitivist vs. interactional approaches}

Within the field of human-computer interaction, Rosalind Picard's cognitivist or "information model" concept of emotion has been criticized by and contrasted with the "post-cognitivist" or "interactional" pragmatist approach taken by Kirsten Boehner and others which views emotion as inherently social.

Picard's focus is human-computer interaction, and her goal for affective computing is to "give computers the ability to recognize, express, and in some cases, 'have' emotions".In contrast, the interactional approach seeks to help "people to understand and experience their own emotions" and to improve computer-mediated interpersonal communication. It does not necessarily seek to map emotion into an objective mathematical model for machine interpretation, but rather let humans make sense of each other's emotional expressions in open-ended ways that might be ambiguous, subjective, and sensitive to context.[example needed]

Picard's critics describe her concept of emotion as "objective, internal, private, and mechanistic". They say it reduces emotion to a discrete psychological signal occurring inside the body that can be measured and which is an input to cognition, undercutting the complexity of emotional experience.

The interactional approach asserts that though emotion has biophysical aspects, it is "culturally grounded, dynamically experienced, and to some degree constructed in action and interaction". Put another way, it considers "emotion as a social and cultural product experienced through our interactions".

\begin{itemize}
\item Boehner, Kirsten; DePaula, Rogerio; Dourish, Paul; Sengers, Phoebe (2007). "How emotion is made and measured". \item Boehner, Kirsten; DePaula, Rogerio; Dourish, Paul; Sengers, Phoebe (2005). "Affection: From Information to Interaction"
\item Hook, Kristina; Staahl, Anna; Sundstrom, Petra; Laaksolahti, Jarmo (2008). "Interactional empowerment"
\end{itemize}

\end{tabular}
\end{center}

\subsection{Body Maps}

 We would like to suggest here, that these representations become more difficult to interpret from one person to another as they are used to document more abstract sensations in the body. When considering that, for instance, two persons may be highlighting the same areas, producing illustrations that look very much alike while trying to describe experience that are entirely different from one another (and visa versa).
 
 \subsubsection{Limitation of Body Maps in HCI}
 Recent literature on soma design has aimed to overcome the temporal limitations of body maps, as “they exist as a snapshot or state representation” \cite{tennent_articulating_2021}. To solve this limitation, Tennent et al. propose the concept of “soma trajectories”: “how a user feels through an interaction, both in body and mind” \cite{tennent_articulating_2021}.

\section{The Intersection of Affective Computing and Social Signal Processing}
\label{lit_reivew:ssp}

We decided devote this subsection to focus on the article by Chanel and Mühl, \textit{Connecting Brains and Bodies: Applying Physiological Computing to Support Social Interaction} \cite{chanel_connecting_2015}, as the content strongly aligns with our research goals. The paper introduces the concept of using physiological data to facilitate affective non-verbal communication, thus combining efforts from Affective Computing and Social Signal Processing.  This begins by acknowledging the use of physiological sensors for traditional emotion recognition tasks and explain that physiological activity can also be related to several social processes, such as empathy. Authors note a study by Levenson and Ruef which relates the physiological linkage between two participants with empathy \cite{levenson_empathy_1992}. Physiological activity can be defined as a social cue, it can convey information about one's emotional state whilst it is not the primary function. For example, blushing as a result of high blood pressure.

Despite physiological data having a strong correlation with affect, authors identify a lack of studies that account for social interaction, and the most studied non-verbal modalities are in fact facial expressions and tone for voice. There are only a few systems that utilise physiological signals to preform social inferences, aside from those from Affective Computing research dedicated to emotional assessment. This can be partially justified considering physiological signals aren't observable. However, technology can be used to display this activity to augment human-human affective communication. 

It's explained how activity from the central nervous system correlates to affective and cognitive states during social interaction, where social emotions include shame, embarrassment, gratitude and admiration and and example of a cognitive signal can be attention. These physiological changes from our autonomic nervous system can be considered highly reliable as the y are very difficult to control and mostly involuntary. 

The article proposes two research directions for using physiological signals for Social Signal Processing contexts. The first is to display social cues to foreign users through physiological data. The second involves analysing physiological activity among multiple users to facilitate collective interaction. 

\section{Conclusion}
\label{lit_review:conclusion}

In this work, we aim to develop and evaluate methods of sharing physiological activity this is abstracted from raw signals and without explicit associations to linguistic descriptors. Through visual, auditory and tactile feedback, we intend to generate new representations of the subject's inner state inferred from analysing physiological signals, which can then be transmitted to foreign users as a means of communication.  

\begin{figure}[htbp]
	\centering
	\includegraphics[width=1.0\textwidth]{Chapters/Figures/Abstracted_Representations.png}
	\caption{Types of Outputs for Representing Physiological Activity}
	\label{fig:Abstracted_Representations}
\end{figure}

In figure \ref{fig:Abstracted_Representations}, we show a simplified plot of how physiological activity can be presented either before or after inference. The leftmost end of the spectrum represents the ideas discussed in section \ref{lit_review:biosignals_sharing}, where live biofeedback depicts the low-level features of the signal. The rightmost end of the spectrum describes what is achieved within Affective Computing systems, where the data is computationally analysed to produce emotional inferences for the user, manifested as a high-level descriptors (e.g fear, joy, surprise, etc...). We intend to demonstrate some kind of middle-ground, where the system is not responsible for determining socio-affective inferences, but we extract particular features from the signal to produce abstracted representations. To achieve this, we propose similar methodologies to those in section \ref{lit_review:biosignals_creativity}, enabling features of the signal to manipulate multimedia parameters. In this case, it's expected for the external user to make socio-affective inferences from the abstracted representation.

Deviating from the methods surveyed in this section, we would like to use this research opportunity to delve into the affordances of contemporary machine learning techniques as a means of generating new representations of biosensor data.  To stay persistent with the thesis objectives, these generative representations should align with the interactional perspective of Affective Computing systems in regards to the non-reductionist principles summarized in section x.x. The aim is not to detect a singular 'right' or 'true' emotion, but rather, to inspire expressive dialogue and emotional reflection [Hook]. At the point of writing, we consider this initiative to be a novel contribution to the field given the impression that machine learning systems are commonly designed for emotion recognition tasks, deriving high-level emotional descriptions from a stream of physiological sensor data.

Our incentive for adopting non-representational machine learning solutions in this context is to exhaust the output possibilities when mapping decoded sensor data of relatively low dimensionality, and embracing high-granularity results that can expose expressive nuances that may not be so salient in the raw data alone.

In addition, we would like to study upon the idea that machine learning systems can improve the level of personalization in affective interactive systems compared to what is normally expected. To achieve this, we would like to develop upon the concepts proposed by Interactive Machine Learning research [http://research.gold.ac.uk/24757/].

% \subsection{Physiological Data and Collective Interaction}

% - HeartBeat: An outdoor pervasive game for children:

% \url{https://www.researchgate.net/publication/221517312_HeartBeat_An_outdoor_pervasive_game_for_children}

% \subsection{Sharing of Biosignals}










