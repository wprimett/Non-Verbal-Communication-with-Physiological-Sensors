%!TEX root = ../template.tex
%%%%%%%%%%%%%%%%%%%%%%%%%%%%%%%%%%%%%%%%%%%%%%%%%%%%%%%%%%%%%%%%%%%
%% chapter1.tex
%% NOVA thesis document file
%%
%% Chapter with introduciton
%%%%%%%%%%%%%%%%%%%%%%%%%%%%%%%%%%%%%%%%%%%%%%%%%%%%%%%%%%%%%%%%%%%
\newcommand{\novathesis}{\emph{novathesis}}
\newcommand{\novathesisclass}{\texttt{novathesis.cls}}


\chapter{Motivation}
\label{cha:introduction}

% \epigraph{
%   This work is licensed under the Creative Commons Attribution-NonCommercial~4.0 International License.
%   To view a copy of this license, visit \url{http://creativecommons.org/licenses/by-nc/4.0/}.
% }

% \section{A Bit of History} % (fold)
% \label{sec:a_bit_of_history}

% The \novathesis\ was originally developed to help MSc and PhD students of the Computer Science and Engineering Department of the Faculty of Sciences and Technology of NOVA University of Lisbon (DI-FCT-NOVA) to write their thesis and dissertations Using \LaTeX.
% %
% These student can easily cope with \LaTeX\ by themselves, and the only need some help in the bootstrap process to make their life easier.

% However, as the template spread out among the students from other degrees at FCT-NOVA, the demand for am easier-to-use template as grown.
% %
% And the template in its current shape aims at answering the expectations of those that, although they are not familiar with programming nor with markup languages, so still feel brave enough to give \LaTeX\ a try and rejoice with the beauty of the texts typeset by this system.

% section a_bit_of_history (end)

\section{Background}
\label{sec:objectives}

\begin{itemize}
  \item Significance of non-verbal communication in social contexts
  \item Most common non-verbal cues are facial expression and tone of voice
  \item Affective Computing proposes the use of physiological activity to infer emotional states. However, these rely on linguistic descriptors
  \item There is an emerging area of work that considers the sharing of physiological activity, proving to be an effective method of enhancing connectedness. However, these studies tend to utilize raw data, disregarding  properties of the signal informative of the subject's socio-affective state.
  \item Aesthetics allow us to convey emotional meaning
\end{itemize}

\section{Research Questions}
\label{sec:research_questions}

\begin{itemize}
  \item How can we represent physiological activity in ways that are non-verbal?
  \item How can aesthetics be Incorporated into visuals, sound and actuation to articulate and express emotional content?
  \item What features can be extracted from each signal to gain information about the subject socio-affective state?
  \item What specific social and emotional cues can be conveyed in a social context?
\end{itemize}

\section{Thesis Structure} % (fold)
\label{sec:structure}

Chapter \ref{cha:technical_concepts} provides a background to the technical and theoretical concepts that form the foundation of thesis. This includes an overview of the different physiological signals commonly used in the field, along with fundamental concepts in Social Signal Processing and Somaestheitic design. 

In Chapter \ref{cha:lit_review}, we comment on some relevant literature covering the following topics, sharing biosignals, biosignals and creative practice, traditional affective computing approaches, and the intersection of social signal processing.

In Chapter \ref{cha:technical_contributions}, we will introduce and examine a set of hardware and software tools that were developed within the duration of the PhD and adopted for the purposes of augmenting nonverbal and collaborative interactions. This will range from specialized wearable devices for physiological data sensing to systems for processing and mapping this incoming data in meaningful ways. This will be followed by a series of applications that have been realised using these tools, described in Chapter \ref{cha:case_studies}. We will then evaluate our use-cases, taking data from user studies.  

\section{Current Challenges}
\label{sec:challenges}

% \section{Limitations} 

% The deliverables of the thesis carry upon theoretical underpinnings grounded in complex movement and bodily practices, namely Yoga and contemporary dance performance. It should be noted that the self-studies carried out during the prototyping in the evaluation stages lack the expert opinions of movement specialists, which would require inclusion professional practitioners. While study subjects were able to admit to holding some valuable experience, at least in complimentary practices, we openly the acknowledge having a limited understanding of the rich intricacies that lie in such specialist areas. 

% In the case of the research actions that called upon specialist uses (e.g Workshops 1.x, CSL Artistic Residency), our results can be praised for interdisciplinary inclusion, for which we benefited from gaining alternative perspectives of the given technologies. That said, this approach without a doubt compromises on longitudidal prospects, imposed by time and budget contingencies. In our case, we found that it was more difficult to obtain structured quantitive data from these user studies, as we embraced more an exploration process of the potential affordances, proceeding to outcomes in the form of interviews, focus groups, etc...

% In reflection of the thesis outcomes, we construct a multi-stage research methodology that first embraces the incorporation of specialist users during an intensive preparation period, supporting longer-term studies that can be safely situated "in-the-wild" when ready.