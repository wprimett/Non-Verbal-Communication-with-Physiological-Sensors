%!TEX root = ../template.tex
%%%%%%%%%%%%%%%%%%%%%%%%%%%%%%%%%%%%%%%%%%%%%%%%%%%%%%%%%%%%%%%%%%%%
%% chapter7.tex
%% NOVA thesis document file
%%
%% Chapter with lots of dummy text
%%%%%%%%%%%%%%%%%%%%%%%%%%%%%%%%%%%%%%%%%%%%%%%%%%%%%%%%%%%%%%%%%%%%

% \chapter{Supplementary Essays: Speculative Developments, Theories and Criticisms of an Emerging Domain}

% \label{cha:supplementary_essays}

% To complete our research outputs, we present a number of perspectives given the speculation and assumption that this is already a matured research space. Some of which are already put forward in the previous sections (\ref{cha:technical_contributions} and \ref{cha:lit_review} in particular), but described more in-depth and unconfined from the current research outcomes of the thesis, which for now can be considered elementary. These are formatted as a series of short essays purposed to shine light upon the socio-political implications of systems that take on our design methodology.

% - Speculative design

% - Claims are not totally supported by our results

% - Should be seen as bridge

% One may wish to interpret this as a brief deviation from the scientific domain to something that is closely related to the humanities. With that said, this should not encourage a dualistic mono-disciplinary perspective between the two, but rather give an adequate margin of expression upon topics that are undeniably crucial yet conventionally undermined in shorter form publications, such as a conference paper or journal article.

\chapter{Results and Discussion}

\section{Study Reflections}

\subsection{Interactional Proxemics}
As is the case with all kinds of non-verbal behaviours, multi-user proxemic interaction is a complex and an inherently ambiguous process. Without constraining the expressive freedoms that would be granted I everyday settings, a real-time biofeedback system should reflect on this. Though, how do we produce meaningful representations when the raw sensor data is desperately limited in dimensionality and temporal scale? Even more so than data from inertial motion sensors  or indeed motion capture.  

We already point out the research scarcity for interactional proxemics, continuously outweighed by studies that are favour the use of asynchronous data analysis, not interfering with the interaction space itself.  This is pretty much customary throughout Behavioural Computing research, though we may raise the question of why the divide is so intensely divided in this case. Does this simply   denote a novel design space that is lacking structural support from academia, or could the experimental reluctance be a consequence of technical challenges specific to the sensing technology?

\subsection{Combining Sensor Data from Multiple Users}

The essence of proxemics implies the use of multimodal information. Our system does not take an explicit measure of orientation, however, we can combine the relative proxemic data from all the sensors in order to distinguish when two or more users are facing towards one another, and register this as periods of mutual gaze. This pluralist gesture reveals a lot about the situation. Generally, we may discern moments of intentional exchange and affirmation, whether that in a comforting or confrontational manner. Of course, this data does not reveal the entire social context by any means, but in contrast to one user staring at the back of the back of another, given the same radial distance, implies a totally different dynamic.

In our system, the sonic representations are designed to emphasise periods of interpersonal gaze. This is detected when the pulse signal from two or more sensors start interfering with one another. 

\section{The Case for Integrating Sensors Data into Movement Practices}

Contemporary Interaction Design (IxD) research groups have incorporated Contact Improvisation routines into their work, enlightening new perceptions of mobility through a collaborative practice that involves the transfer of body weight between partners \cite{bomba_somacoustics_2019, barrero_gonzalez_dance_2019}. Where the principal activity presumes that intimate space is to be co-occupied and that touch is freely permitted, in some cases using intense pressure, such practices emphasise the importance of consent and appropriating of physical contact, though several practitioners have come forward to challenge this idealism \cite{tai_exploring_2017,beaulieux_how_2019}.

In the our initial pursuit towards integrating established somatic practices to the domain of sensory technologies, a two month research residency commenced with KTH, Royal Institute of Technology.The proposed outcome for this collaboration was ultimately to develop an Interactive Machine learning framework that models specific physiological characteristics informed by Contact e. This involved a small series of Contact Improvisation session, guided by an expert practitioner, serving as a somatic connoisseur. Between each activity, would highlight some of the aesthetically informed sensations that occurred, and try to develop sensor-actuation couplings that would mirror the inter-user movement qualities.

On a personal account, the intermediate CI workshops did not enlighten the prolific research opportunities that were naively anticipated up until the research residency. First off, we were not so clear as to the major role of technological intervention. For example, to guide movement patterns in substitute of a facilitator, or exaggerate tactile sensation that occur. In terms of inclusion, the experiential gap between the facilitator and the researchers felt disruptive to personal exploration. Even given a successful digitisation with embodied sensors, how would this experience be distributed outside of the lab or even studio? Without dwelling much further, this segment concludes with the a call to new research directions that realised throughout our research outputs. Ultimately, we are interested in systems capable to instigate interaction without the necessity of explicit instructions. To essentially allow participant to experience the interactive artefact through experimentation, preferably shared with others.

\section{Limitations}
\label{sec:limitations}

\subsection{Self Studies and Non-lab Experimentation Settings}

The deliverables of the thesis carry upon theoretical underpinnings grounded in complex movement and bodily practices, namely Yoga and contemporary dance performance. It should be noted that the self-studies carried out during the prototyping in the evaluation stages lack the expert opinions of movement specialists, which would require inclusion professional practitioners. While study subjects were able to admit to holding some valuable experience, at least in complimentary practices, we openly the acknowledge having a limited understanding of the rich intricacies that lie in such specialist areas.

In the case of the research actions that called upon specialist uses (e.g Workshops 1.x, CSL Artistic Residency), our results can be praised for interdisciplinary inclusion, for which we benefited from gaining alternative perspectives of the given technologies. That said, this approach without a doubt compromises on longitudinal prospects, imposed by time and budget contingencies. In our case, we found that it was more difficult to obtain structured quantities data from these user studies, as we embraced more an exploration process of the potential affordances, proceeding to outcomes in the form of interviews, focus groups, etc…

We found many difficulties to come against the scientific rigour seen in clinical trials, and the possibility to validate numerical findings. In reflection of the thesis outcomes, we construct a multi-stage research methodology that first embraces the incorporation of specialist users during an intensive preparation period, supporting longer-term studies that can be safely situated “in-the-wild” when ready.

In public space, participants are not being observed, their behaviour, even if not aligned with the study protocol, is authentic. They are responsible for using a system according to their personal intuition, not necessarily what the designer intended. Unlike lab trials, public space experimentation lends itself to unforeseeable events. Every encounter is unique in a way that cannot be perfectly repeated. 


